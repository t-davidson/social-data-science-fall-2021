% Options for packages loaded elsewhere
\PassOptionsToPackage{unicode}{hyperref}
\PassOptionsToPackage{hyphens}{url}
%
\documentclass[
  10pt,
]{article}
\usepackage{amsmath,amssymb}
\usepackage[]{mathpazo}
\usepackage{ifxetex,ifluatex}
\ifnum 0\ifxetex 1\fi\ifluatex 1\fi=0 % if pdftex
  \usepackage[T1]{fontenc}
  \usepackage[utf8]{inputenc}
  \usepackage{textcomp} % provide euro and other symbols
\else % if luatex or xetex
  \usepackage{unicode-math}
  \defaultfontfeatures{Scale=MatchLowercase}
  \defaultfontfeatures[\rmfamily]{Ligatures=TeX,Scale=1}
\fi
% Use upquote if available, for straight quotes in verbatim environments
\IfFileExists{upquote.sty}{\usepackage{upquote}}{}
\IfFileExists{microtype.sty}{% use microtype if available
  \usepackage[]{microtype}
  \UseMicrotypeSet[protrusion]{basicmath} % disable protrusion for tt fonts
}{}
\makeatletter
\@ifundefined{KOMAClassName}{% if non-KOMA class
  \IfFileExists{parskip.sty}{%
    \usepackage{parskip}
  }{% else
    \setlength{\parindent}{0pt}
    \setlength{\parskip}{6pt plus 2pt minus 1pt}}
}{% if KOMA class
  \KOMAoptions{parskip=half}}
\makeatother
\usepackage{xcolor}
\IfFileExists{xurl.sty}{\usepackage{xurl}}{} % add URL line breaks if available
\IfFileExists{bookmark.sty}{\usepackage{bookmark}}{\usepackage{hyperref}}
\hypersetup{
  pdfauthor={Dr.~Thomas Davidson},
  hidelinks,
  pdfcreator={LaTeX via pandoc}}
\urlstyle{same} % disable monospaced font for URLs
\usepackage[margin=1in]{geometry}
\usepackage{graphicx}
\makeatletter
\def\maxwidth{\ifdim\Gin@nat@width>\linewidth\linewidth\else\Gin@nat@width\fi}
\def\maxheight{\ifdim\Gin@nat@height>\textheight\textheight\else\Gin@nat@height\fi}
\makeatother
% Scale images if necessary, so that they will not overflow the page
% margins by default, and it is still possible to overwrite the defaults
% using explicit options in \includegraphics[width, height, ...]{}
\setkeys{Gin}{width=\maxwidth,height=\maxheight,keepaspectratio}
% Set default figure placement to htbp
\makeatletter
\def\fps@figure{htbp}
\makeatother
\setlength{\emergencystretch}{3em} % prevent overfull lines
\providecommand{\tightlist}{%
  \setlength{\itemsep}{0pt}\setlength{\parskip}{0pt}}
\setcounter{secnumdepth}{-\maxdimen} % remove section numbering
\linespread{1.05}
\ifluatex
  \usepackage{selnolig}  % disable illegal ligatures
\fi

\title{Social Data Science\\
Rutgers University\\
~\\
\hspace*{0.333em}Syllabus}
\author{Dr.~Thomas Davidson}
\date{Fall 2021}

\begin{document}
\maketitle

\hypertarget{contact-and-logistics}{%
\section{CONTACT AND LOGISTICS}\label{contact-and-logistics}}

E-mail: \texttt{thomas.davidson@rutgers.edu} or Canvas message.

Website:
\texttt{https://github.com/t-davidson/social-data-science-fall-2021} and
Canvas.

Class meetings: MW 3-4:20 p.m, Loree Classroom Building - Room 020.

Office hours: W 4:30-5:30 p.m, Davison Hall - Room 109 or by
appointment.

\hypertarget{course-description}{%
\section{COURSE DESCRIPTION}\label{course-description}}

This course introduces students to the growing field of computational
social science. Students will learn to collect and critically analyze
social data using a range of techniques including natural language
processing, machine learning, and agent-based modeling. We will discuss
how these techniques are used by social scientists and consider the
ethical implications of big data and artificial intelligence. Students
will complete homework assignments involving coding in the R programming
language to analyze several different datasets and will complete a group
project to create a web-based application for data analysis and
visualization.

\hypertarget{prerequisites}{%
\section{PREREQUISITES}\label{prerequisites}}

\emph{Data 101} or equivalent. Enrolled students \emph{must} have
experience writing basic programs in a general purpose programming
language, e.g.~R, Python, Java, C.

We will review the fundamentals for programming and data science in R in
weeks 1-3.

\hypertarget{assessment}{%
\section{ASSESSMENT}\label{assessment}}

\begin{itemize}
\tightlist
\item
  10\% Class participation
\item
  60\% Homework assignments (3 x 20\%)
\item
  30\% Group project (R Shiny app and write-up)
\end{itemize}

\hypertarget{readings}{%
\section{READINGS}\label{readings}}

Most of the readings will consist of chapters from the textbooks listed
below. These readings are intended to build familiarity with key
concepts and programming skills. Some weeks there will be an additional
reading to highlight how data science techniques are used in empirical
social scientific research.

\hypertarget{textbooks}{%
\subsection{Textbooks}\label{textbooks}}

\emph{* indicates a required text. All required texts and useful
references are available for free online on the listed websites.}

\begin{itemize}
\tightlist
\item
  *Matthew Salganik. 2017. \emph{Bit by Bit}. Princeton University
  Press. \url{https://www.bitbybitbook.com/en/1st-ed/preface/}
\item
  *Wickham, Hadley, and Garrett Grolemund. 2016. \emph{R for Data
  Science: Import, Tidy, Transform, Visualize, and Model Data}.
  (\emph{R4DS}). O'Reilly Media, Inc. \url{https://r4ds.had.co.nz/}
\item
  *Silge, Julia, and David Robinson. 2017. \emph{Text Mining with R: A
  Tidy Approach.} O'Reilly Media.
  \url{https://www.tidytextmining.com/dtm.html}.
\item
  Healy, Kieran. 2018. \emph{Data Visualization: A Practical
  Introduction}. Princeton University Press. \url{https://socviz.co/}
\end{itemize}

\hypertarget{resources}{%
\section{RESOURCES}\label{resources}}

The course will be organized using two different tools, Github and
Canvas. Canvas will be used for class communication, short quizzes, and
for scheduling. Github Classroom will be used for the submission of
assignments.

\hypertarget{course-policies}{%
\section{COURSE POLICIES}\label{course-policies}}

The Rutgers Sociology Department strives to create an environment that
supports and affirms diversity in all manifestations, including race,
ethnicity, gender, sexual orientation, religion, age, social class,
disability status, region/country of origin, and political orientation.
This class will be a space for tolerance, respect, and mutual dialogue.
Students must abide by the Code of Student Conduct at all times,
including during lectures and in participation online.

All students must abide by the university's Academic Integrity Policy.
Violations of academic integrity will result in disciplinary action.

In accordance with University policy, if you have a documented
disability and require accommodations to obtain equal access in this
course, please contact me during the first week of classes. Students
with disabilities must be registered with the Office of Student
Disability Services and must provide verification of their eligibility
for such accommodations.

I will also be making additional accommodations due to the COVID-19
pandemic. If you or your family are affected in any way that impedes
your ability to participate in this course, please contact me as soon as
you can so that we can make necessary arrangements.

\hypertarget{course-outline}{%
\section{COURSE OUTLINE}\label{course-outline}}

\textbf{\emph{This outline is tentative and subject to change.}}

\hypertarget{week-1-91-wednesday-only}{%
\subsection{Week 1, 9/1 (Wednesday
only)}\label{week-1-91-wednesday-only}}

\hypertarget{introduction-to-social-data-science}{%
\subsubsection{Introduction to social data
science}\label{introduction-to-social-data-science}}

\emph{Readings}

\begin{itemize}
\tightlist
\item
  \emph{Bit by Bit}, C1
\item
  \emph{R4DS}: C1 \& 27 {[}Note: Chapter numbers correspond to the
  online book; physical book numbers are different{]}
\end{itemize}

\hypertarget{week-2-98-wednesday-only}{%
\subsection{Week 2, 9/8 (Wednesday
only)}\label{week-2-98-wednesday-only}}

\hypertarget{data-structures-in-r}{%
\subsubsection{Data structures in R}\label{data-structures-in-r}}

\emph{Readings}

\begin{itemize}
\tightlist
\item
  \emph{R4DS}: C2,4, skim 20.
\end{itemize}

\hypertarget{week-3-913-915}{%
\subsection{Week 3, 9/13 \& 9/15}\label{week-3-913-915}}

\hypertarget{programming-fundamentals}{%
\subsubsection{Programming
fundamentals}\label{programming-fundamentals}}

\emph{Readings}

\begin{itemize}
\tightlist
\item
  Monday, \emph{R4DS}: C17-19, 21.
\item
  Wednesday, \emph{R4DS}: C5, 9, 10, 13.
\end{itemize}

\hypertarget{assignment-1-released-using-r-for-data-science.}{%
\subsection{\texorpdfstring{\emph{Assignment 1 released: Using R for
Data
Science.}}{Assignment 1 released: Using R for Data Science.}}\label{assignment-1-released-using-r-for-data-science.}}

\hypertarget{week-4-920-922}{%
\subsection{Week 4, 9/20 \& 9/22}\label{week-4-920-922}}

\hypertarget{data-collection-i-collecting-data-using-application-programming-interfaces}{%
\subsubsection{Data Collection I: Collecting data using Application
Programming
Interfaces}\label{data-collection-i-collecting-data-using-application-programming-interfaces}}

\emph{Readings}

\begin{itemize}
\tightlist
\item
  Monday, \emph{Bit by Bit}, C2
\item
  Wednesday, \emph{R4DS}: C3
\end{itemize}

\hypertarget{week-5-927-929}{%
\subsection{Week 5, 9/27 \& 9/29}\label{week-5-927-929}}

\hypertarget{data-collection-ii-scraping-data-from-the-web}{%
\subsubsection{Data Collection II: Scraping data from the
web}\label{data-collection-ii-scraping-data-from-the-web}}

\emph{Readings}

\begin{itemize}
\tightlist
\item
  Monday, \emph{Bit by Bit}, C6
\item
  Wednesday, \emph{R4DS}: C14, 16
\end{itemize}

\emph{Recommended}

\begin{itemize}
\tightlist
\item
  Fiesler, Casey, Nate Beard, and Brian C Keegan. 2020. ``No Robots,
  Spiders, or Scrapers: Legal and Ethical Regulation of Data Collection
  Methods in Social Media Terms of Service.'' In \emph{Proceedings of
  the Fourteenth International AAAI Conference on Web and Social Media},
  187--96.
\end{itemize}

\hypertarget{week-6-104-106}{%
\subsection{Week 6, 10/4 \& 10/6}\label{week-6-104-106}}

\hypertarget{data-collection-iii-online-experiments-and-surveys}{%
\subsubsection{Data Collection III: Online experiments and
surveys}\label{data-collection-iii-online-experiments-and-surveys}}

\hypertarget{assignment-2-collecting-and-storing-data-released.}{%
\subsection{\texorpdfstring{\emph{Assignment 2: Collecting and storing
data
released.}}{Assignment 2: Collecting and storing data released.}}\label{assignment-2-collecting-and-storing-data-released.}}

\emph{Readings}

\begin{itemize}
\tightlist
\item
  R Shiny tutorial: \url{https://shiny.rstudio.com/tutorial/}
\item
  \emph{Bit by Bit}, C3-5
\end{itemize}

\hypertarget{week-7-1011-1012}{%
\subsection{Week 7, 10/11 \& 10/12}\label{week-7-1011-1012}}

\hypertarget{natural-language-processing-i-the-vector-space-model}{%
\subsubsection{Natural Language Processing I: The vector-space
model}\label{natural-language-processing-i-the-vector-space-model}}

\emph{Readings}

\begin{itemize}
\tightlist
\item
  \emph{Text Mining with R}, C1 \& 3
\end{itemize}

\emph{Recommended}

\begin{itemize}
\tightlist
\item
  Evans, James, and Pedro Aceves. 2016. ``Machine Translation: Mining
  Text for Social Theory.'' \emph{Annual Review of Sociology} 42 (1):
  21--50. \url{https://doi.org/10.1146/annurev-soc-081715-074206}.
\end{itemize}

\hypertarget{week-8-1018-1020}{%
\subsection{Week 8, 10/18 \& 10/20}\label{week-8-1018-1020}}

\hypertarget{natural-language-processing-ii-word-embeddings}{%
\subsubsection{Natural Language Processing II: Word
embeddings}\label{natural-language-processing-ii-word-embeddings}}

\emph{Readings}

\begin{itemize}
\tightlist
\item
  \emph{Text Mining with R}: C5.
\item
  Hvitfeldt, Emil and Julia Silge. 2020 \emph{Supervised Machine
  Learning for Text Analysis in R.} Chapter 5:
  \url{https://smltar.com/embeddings.html}.
\end{itemize}

\emph{Recommended}

\begin{itemize}
\tightlist
\item
  Kozlowski, Austin, Matt Taddy, and James Evans. 2019. ``The Geometry
  of Culture: Analyzing the Meanings of Class through Word Embeddings.''
  \emph{American Sociological Review}, September, 000312241987713.
  \url{https://doi.org/10.1177/0003122419877135}.
\end{itemize}

\hypertarget{week-9-1025-1027}{%
\subsection{Week 9, 10/25 \& 10/27}\label{week-9-1025-1027}}

\hypertarget{natural-language-processing-iii-topic-models}{%
\subsubsection{Natural Language Processing III: Topic
models}\label{natural-language-processing-iii-topic-models}}

\hypertarget{assignment-3-natural-language-processing-released.}{%
\subsection{\texorpdfstring{\emph{Assignment 3: Natural language
processing
released.}}{Assignment 3: Natural language processing released.}}\label{assignment-3-natural-language-processing-released.}}

\emph{Readings}

\begin{itemize}
\tightlist
\item
  \emph{Text Mining with R}: C6.
\item
  Mohr, John, and Petko Bogdanov. 2013. ``Introduction---Topic Models:
  What They Are and Why They Matter.'' \emph{Poetics} 41 (6): 545--69.
  \url{https://doi.org/10.1016/j.poetic.2013.10.001}.
\end{itemize}

\emph{Recommended}

\begin{itemize}
\tightlist
\item
  Roberts, Margaret, Brandon M. Stewart, Dustin Tingley, Christopher
  Lucas, Jetson Leder-Luis, Shana Kushner Gadarian, Bethany Albertson,
  and David Rand. 2014. ``Structural Topic Models for Open-Ended Survey
  Responses: Structural Topic Models for Survey Responses.''
  \emph{American Journal of Political Science} 58 (4): 1064--82.
  \url{https://doi.org/10.1111/ajps.12103}.
\end{itemize}

\hypertarget{week-10-111-113}{%
\subsection{Week 10, 11/1 \& 11/3}\label{week-10-111-113}}

\hypertarget{machine-learning-i-prediction-and-explanation}{%
\subsubsection{Machine Learning I: Prediction and
explanation}\label{machine-learning-i-prediction-and-explanation}}

\emph{Readings}

\begin{itemize}
\tightlist
\item
  Molina, Mario, and Filiz Garip. 2019. ``Machine Learning for
  Sociology.'' \emph{Annual Review of Sociology} 45: 27--45.
\end{itemize}

\hypertarget{week-11-118-1110}{%
\subsection{Week 11, 11/8 \& 11/10}\label{week-11-118-1110}}

\hypertarget{machine-learning-ii-text-classification}{%
\subsubsection{Machine learning II: Text
classification}\label{machine-learning-ii-text-classification}}

\emph{Readings}

\begin{itemize}
\tightlist
\item
  Hanna, Alex. 2013. ``Computer-Aided Content Analysis of Digitally
  Enabled Movements.'' \emph{Mobilization: An International Quarterly}
  18 (4): 367--388.
\end{itemize}

\emph{Recommended}

\begin{itemize}
\tightlist
\item
  Barberá, Pablo, Amber E. Boydstun, Suzanna Linn, Ryan McMahon, and
  Jonathan Nagler. 2020. ``Automated Text Classification of News
  Articles: A Practical Guide.'' \emph{Political Analysis}, June, 1--24.
  \url{https://doi.org/10.1017/pan.2020.8}.
\end{itemize}

\hypertarget{week-12-1115-1117}{%
\subsection{Week 12, 11/15 \& 11/17}\label{week-12-1115-1117}}

\hypertarget{machine-learning-iii-challenges}{%
\subsubsection{Machine learning III:
Challenges}\label{machine-learning-iii-challenges}}

\emph{Readings}

\begin{itemize}
\tightlist
\item
  Salganik, Matthew, Ian Lundberg, Alexander Kindel, et al.~2020.
  ``Measuring the Predictability of Life Outcomes with a Scientific Mass
  Collaboration.'' \emph{Proceedings of the National Academy of
  Sciences}.
\item
  Buolamwini, Joy, and Timnit Gebru. 2018. ``Gender Shades:
  Intersectional Accuracy Disparities in Commercial Gender
  Classification.'' In \emph{Proceedings of Machine Learning Research},
  81:1--15.
\end{itemize}

\hypertarget{week-13-1122-no-class-wednesday-for-thanksgiving}{%
\subsection{Week 13, 11/22 (No class Wednesday for
Thanksgiving)}\label{week-13-1122-no-class-wednesday-for-thanksgiving}}

\hypertarget{machine-learning-iv-image-classification}{%
\subsubsection{Machine learning IV: Image
classification}\label{machine-learning-iv-image-classification}}

\emph{Readings}

\begin{itemize}
\tightlist
\item
  Torres, Michelle, and Francisco Cantú. 2021. ``Learning to See:
  Convolutional Neural Networks for the Analysis of Social Science
  Data.'' \emph{Political Analysis}, April, 1--19.
  \url{https://doi.org/10.1017/pan.2021.9}.
\item
  Gebru, Timnit, Jonathan Krause, Yilun Wang, Duyun Chen, Jia Deng, Erez
  Lieberman Aiden, and Li Fei-Fei. 2017. ``Using Deep Learning and
  Google Street View to Estimate the Demographic Makeup of Neighborhoods
  across the United States.'' \emph{Proceedings of the National Academy
  of Sciences} 114 (50): 13108--13.
  \url{https://doi.org/10.1073/pnas.1700035114}.
\end{itemize}

\hypertarget{week-14-1129-111}{%
\subsection{Week 14, 11/29 \& 11/1}\label{week-14-1129-111}}

\hypertarget{simulation-and-agent-based-models}{%
\subsubsection{Simulation and agent-based
models}\label{simulation-and-agent-based-models}}

\emph{Readings}

\begin{itemize}
\tightlist
\item
  Macy, Michael, and Robert Willer. 2002. ``From Factors to Factors:
  Computational Sociology and Agent-Based Modeling.'' \emph{Annual
  Review of Sociology} 28 (1): 143--66.
  \url{https://doi.org/10.1146/annurev.soc.28.110601.141117}.
\item
  \url{https://cran.r-project.org/web/packages/shinySIR/vignettes/Vignette.html}
\end{itemize}

\hypertarget{week-15-126-127}{%
\subsection{Week 15, 12/6 \& 12/7}\label{week-15-126-127}}

\hypertarget{presentations}{%
\subsubsection{Presentations}\label{presentations}}

\hypertarget{final-projects-due-1216-at-5pm}{%
\subsection{\texorpdfstring{\emph{Final projects due 12/16 at
5pm}}{Final projects due 12/16 at 5pm}}\label{final-projects-due-1216-at-5pm}}

\end{document}
